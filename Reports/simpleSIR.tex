
\documentclass[12pt,letterpaper]{article}
\usepackage{amsmath}
\usepackage{amsfonts}
\usepackage[figurewithin=section,tablewithin=section]{caption}
\usepackage[usenames,dvipsnames]{color}
\usepackage{graphicx}
\usepackage{longtable}
\usepackage{rotating}
\usepackage{booktabs}
\usepackage[bibencoding=utf8, citestyle=authoryear,bibstyle=authoryear,maxbibnames=99]{biblatex}
%\addbibresource{~/Projects/MyBibtex.bib}

\bibliography{/home/jsibert/Projects/MyBibtex.bib,/home/jsibert/MendeleyBibTex/library.bib}


\usepackage[pdftex,bookmarks=false]{hyperref}
\hypersetup{pdfauthor={John Sibert}
            pdfsubject={variable infection and mortality rate estimates}
            pdftitle={Simple SIR Statistical Model}
            pdfkeywords={COVID-19, SIR Model, random effects, TMB}
            }%

\newcommand\doublespacing{\baselineskip=1.6\normalbaselineskip}
\newcommand\singlespacing{\baselineskip=1.0\normalbaselineskip}

\title{Estimating short term trends in infection and mortality rates
during the Covid 19 Epidemic}

\author{
John Sibert\thanks{johnrsibert@gmail.com}\\
Joint Institute of Marine and Atmospheric Research\\
University of Hawai`i at M\={a}noa\\
Honolulu, HI  96822 U.S.A.\\[0.125in]
\date{\today}
}

\begin{document}

\maketitle

\doublespacing

\section*{Introduction}

The sudden advent of the COVID-19 pandemic provoked many political
jurisdictions to advise people to ``shelter in place'' and to practice
``social distancing''. If this advice has been effective it should be
possible to detect the effects of the advice by comparing changes
numbers of infected people and perhaps changes in the infection rates
over time and
between areas. The SIR models of epidemic spread divide the affected
population into three compartments: 
Susceptible, Infectious and Recovered.
SIR models are
usually expressed as coupled ordinary differential equations,
\begin{eqnarray}
\label{eqn:SIR}
\frac{dS}{dt} &=& -\beta\frac{IS}{N} - \mu S\\
\frac{dI}{dt} &=& \hphantom{-}\beta\frac{IS}{N} - \mu I -\gamma I\\
\frac{dR}{dt} &=&  -\mu R +\gamma I
\end{eqnarray}
where $N$ is the population size, $\beta$ is the instantaneous
infection rate ($[t^{-1}]$), $\mu$ is the instantaneous mortality rate
($[t^{-1}]$),  and $\gamma$ is the instantaneous recovery rate
($[t^{-1}]$).  

Unfortunately, few data sets include data for each of
these compartments. 
The New York Times' ``historical'' data
(https://github.com/nytimes/covid-19-data/) is an
easily accessible and source of data. These
data comprise daily totals of ``cases'' and ''deaths'' for each county
in the United States. I assume that the data included as ``cases'' are
a reasonable approximations of the Infectious compartment ($I$) in a SIR model. 
There is simply no credible information on either the Susceptible or
the Recovered compartments.

\section*{Model Structure}
I make some simplifying assumptions. (1) The entire population is
susceptible so that $S/N = 1$. (2) Over the short term, the size of the
Susceptible compartment does not change, $\frac{dS}{dt} = 0$.
(3) People who recover from a COVID-19 infection return to the Susceptible
compartment, eliminating the Recovered compartment. 
With these assumptions, and the addition of a ``deaths''
compartment the simplified SIR model is
\begin{eqnarray}
\label{eqn:sSIR}
\frac{dI}{dt} &=&  \beta I - \mu I -\gamma I\\
\frac{dD}{dt} &=& \mu I
\end{eqnarray}
and has state variables that might be fitted to available observations.

The data available during the initial stages of the COVID-19 pandemic
contain measurement errors of various types.
Definitions and methods of detecting and reporting the numbers of
infected persons vary between political jurisdictions (or
``geographies'' in the parlance of the New York Times) and may also
change with time.
Comparable uncertainties also occur in reporting of deaths caused
by COVID-19 infection.
There is additional variability in the biosocial
processes that mediate disease transmission.

State-space models separate variability in the biosocial
processes in the system (transition model)
from errors in observing features of interest
in the system (observation model).
(See \cite{Harvey1990}).

The general form of a state-space transition model is
\begin{equation}
\alpha_t=T(\alpha_{t-1}) + \eta_t
\end{equation}
where $\alpha_t$ is the state at time $t$ and 
the function $T$ embodies the dynamics mediating the
development of the state at time $t$ from the state at the previous
time with random process error, $\eta_t$.

The transition model is constructed from finite difference
approximations of equation (\ref{eqn:sSIR}) with associated log-normal
random errors.
\begin{eqnarray}
\label{eqn:sSIRfd}
I_t &=& I_{t-\Delta t}\big(1+\Delta t(\beta_{t-\Delta t} - \mu_{t-\Delta t}
- \gamma)\big)e^{\eta_t}\\
D_t &=& \big(D_{t-\Delta t} + \Delta t \mu_{t-\Delta t}I_{t-\Delta
t}\big)e^{\eta_t}
\end{eqnarray}
where $\eta$ is a log-normal random deviate, $\eta\sim
N(0,\sigma_\eta)$, representing temporal variability in the biosocial
factors that mediate the spread of the pandemic. I have no particular
justification, beyond the parsimony principle, for the assumption that
the variance of these two processes, $\sigma_\eta$, should be the
same.

One approach to modeling time-dependent rates of infection and
mortality, $\beta$ and $\mu$, is to treat them as random effects
(\cite{Skaug2006}). Random effects are appropriate if repeating a time
series of observations would not yield the same outcome as the initial
observations. Random effects would also be important when observing
the same process in two different areas. I model the  $\beta$ and
$\mu$ time series as log-normal random walks. I assume that
\begin{eqnarray}
\log\beta_t &=& \log\beta_{t-\Delta t}+\varepsilon;\quad \varepsilon\sim 
N(0,\sigma_\beta)\\
\log\mu_t &=& \log\mu_{t-\Delta t}+\varrho;\quad \varrho\sim
N(0,\sigma_\mu)
\end{eqnarray}

The general form of the state-space observation model is
\begin{equation}
x_t = O(\alpha_t) + \varepsilon_t
\end{equation}
where the function $O$ describes the measurement process with
error $\varepsilon$ in observing the state.

I applied different error models for cases and
deaths. The observation model for cases is a simple log-normal error
\begin{equation}
\log\varphi_t = \bigg(\log\frac{1}{\sqrt{2\pi\sigma^2_I}} -\Bigl(\frac{\log
I_t-\log\widehat{I}_t}{\sigma_I}\Bigr)^2\bigg)\\
\end{equation}
where $I$ is the observed number of cases and $\widehat{I}$ is the
number of cases predicted by equation~\ref{eqn:sSIRfd}.

The deaths time-series inevitably contains a
substantial number of recorded zeros. 
Not all those afflicted by COVID-19 have died; there are far fewer
deaths than infections. In addition,
the observed time series for both $I$ and $D$ begins at the first recorded
case. The first recorded death occurs several days or weeks after the
first recorded case. 
The observation model for deaths accommodates observed zeroes by
assuming to be ``zero-inflated'' log normal likelihood given by
\begin{equation}
  \log \varepsilon_t = \left\{
    \begin{array}{r@{\;:\quad}l}
       D_t > 0 &
(1-p_0)\cdot\bigg(\log\frac{1}{\sqrt{2\pi\sigma^2_D}}
          -\Bigl(\frac{\log D_t-\log\widehat{D}_t}{\sigma_D}\Bigr)^2\bigg)\\
       D_t = 0 & p_0 \cdot\log \frac{1}{\sqrt{2\pi\sigma^2_D}}\\
    \end{array}
  \right.
\end{equation}
where $D$ is the observed number of deaths,
$\widehat{D}$ is the number of deaths predicted by
equation~\ref{eqn:sSIRfd}, 
and $p_0$ is the proportion of observed deaths equal to zero.

\cite{Sibert2017,Nielsen2014b}


Model parameters are estimated by
maximizing the joint likelihood of the process errors, observation
errors, and random effects.
\begin{equation}
\label{eqn:likelihood}
L(\theta,\alpha,x)=
\prod^m_{t=2}\big[\phi\big(\alpha_t-T(\alpha_{t-1}), \Sigma_\eta\big)\big]\cdot
\prod^m_{t=1}\big[\phi\big(x_t-O(\alpha_t),
\Sigma_\varepsilon\big)\big]
\end{equation}
where $m$ is the number of days elapsed since the first recorded case,
$x_t$ is the vector of daily observations of cases and deaths,
$\alpha_t$ is the vector of the daily calculations of the state
variables and random effects,
and $\theta$ 
is a vector of model parameters (Table~\ref{tab:allvars1}).
The model is implemented in TMB (\cite{TMB0000}).

\url{https://github.com/johnrsibert/SIR-Models}.



\begin{table}
\caption{Complete list of model variables for the simple SIR model.
There are two state variables computed from the of estimated
parameters and random effects.
There are one two random effects for each time step in the model
depending on model configuration.
There are six or seven estimated parameters depending on model
configuration. 
}
\label{tab:allvars1}
\begin{center}
\begin{tabular}{ll}
\hline
Variable & Definition\\
\hline
\hline
       & {\it State variables:}\\
I      & Number of infected individuals or ``cases''\\
D      & Number of deaths\\
       & {\it Random effects:}\\
$\beta_t$ & Infection rate\\
$\mu_t$   & Mortality rate\\
       & {\it Estimated parameters:}\\
$\gamma$   & Recovery rate of infected individuals\\
$\mu$      & Mortality rate of infected individuals\\
$\sigma_I$ & Infectious compartment estimation standard deviation\\
$\sigma_D$ & Deaths compartment estimation standard deviation\\
$\sigma_\eta$ & Standard deviation of Infectious and Deaths compartments\\
$\sigma_\beta$ & Standard deviation of infection rate random walk\\
$\sigma_\mu$ & Standard deviation of mortality rate random walk\\
\hline
\end{tabular}
\end{center}
\end{table}

\section*{Results}

Whether the available data are sufficiently informative to enable
estimation of the model parameters is a critical aspect of the
evaluation of any statistical model.
The speed at which the COVID-19 pandemic spread during the first
quarter of
2020 means that the length of the time series nearly doubled during
the development of this model. Consequently, the capability of the
model seemed to improve during the development period.
Whether the improvement is
attributable to changes in model structure or to the increase in the
length of the time series is unclear. This ambiguity influenced the
development of the model.

The preliminary model presented here has only one random effect
infection rate, $\beta_t$, and the mortality rate, $\mu$, is
considered constant over time. There are thus seven estimated
parameters.      

\clearpage
\printbibliography[title=References]


\end{document}

