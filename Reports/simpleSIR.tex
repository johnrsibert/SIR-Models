
\documentclass[12pt,letterpaper]{article}
\usepackage{amsmath}
\usepackage{amsfonts}
\usepackage[figurewithin=section,tablewithin=section]{caption}
\usepackage[usenames,dvipsnames]{color}
\usepackage{graphicx}
\usepackage{longtable}
\usepackage{rotating}
\usepackage{booktabs}
%\usepackage[citestyle=authoryear,bibstyle=authoryear,maxbibnames=99]{biblatex}
%\addbibresource{~/MendeleyBibTex/library.bib}


\usepackage[pdftex,bookmarks=false]{hyperref}
\hypersetup{pdfauthor={John Sibert}
            pdfsubject={variable infection and mortality rate estimates}
            pdftitle={Simple SIR Statistical Model}
            pdfkeywords={COVID-19, SIR Model, random effects, TMB}
            }%

\newcommand\doublespacing{\baselineskip=1.6\normalbaselineskip}
\newcommand\singlespacing{\baselineskip=1.0\normalbaselineskip}

\title{Estimating short term trends in infection and mortality rates
during the Covid 19 Epidemic}

\author{
John Sibert\thanks{johnrsibert@gmail.com}\\
Joint Institute of Marine and Atmospheric Research\\
University of Hawai`i at M\={a}noa\\
Honolulu, HI  96822 U.S.A.\\[0.125in]
\date{\today}
}

\begin{document}

\maketitle

\doublespacing

\section*{Introduction}

SIR models of epidemic spread divide the population into three
compartments: Susceptible, Infectious and Recovered.
Unfortunately, there are few data sets that include data from each of
these three compartments. Furthermore, the data that are available
probably contain measurement errors of various types. SIR models are
often expressed as three coupled ordinary differential equations,
\begin{eqnarray}
\label{eqn:SIR}
\frac{dS}{dt} &=& -\beta\frac{IS}{N} - \mu S\\
\frac{dI}{dt} &=&  \beta\frac{IS}{N} - \mu I -\gamma I\\
\frac{dR}{dt} &=& \gamma I - \mu R 
\end{eqnarray}
where $N$ is the population size, $\beta$ is the instantaneous
infection rate ($t^{-1}$), $\mu$ is the instantaneous death rate
($t^{-1}$),  and $\gamma$ is the instantaneous recovery rate ($t^{-1}$).  

The New York Times ``historical'' data
(https://github.com/nytimes/covid-19-data/) is the
most easily accessible and source of data. It is updated daily. These
data comprise daily totals of ``cases'' and ''deaths'' for each county
in the United States. I assume that the data included as ``cases'' are
a reasonable estimate of the Infectious complarment ($I$) in a SIR model. 
There is
simply no information of either Susceptible or Recovered compartments.

So I make some simplifying assumptions. The entire population is
susceptible so that $S/N = 1$. Over the short term, the size of the
Susceptible compartment does not change, $\frac{dS}{dt} = 0$.
People who recover from a COVID-19 infection return to the Susceptible
compartment. With these assumptions, and the addition of a ``dead''
compartment the simplified SIR model becomes
\begin{eqnarray}
\label{eqn:sSIR}
\frac{dI}{dt} &=&  \beta I - \mu I -\gamma I\\
\frac{dD}{dt} &=& \mu I
\end{eqnarray}
and has state variables that might be fitted to observations.

It is widely accepted that there are errors in the data. Methods and
definitions vary
between political jurisdictions (or ``geographies'' in the parlance of
the New York Times). There is also variability in the biosocial
processes that mediate disease transmission.
I take a state-space approach to fitting the model to data by
separating process error due to biosocial factors from observation
error. 

State-space models separate variability in the biosocial
processes in the system (transition model)
from errors in observing features of interest
in the system (observation model).

The general form of the {\bf transition model} is
\begin{displaymath}
\alpha_t=T(\alpha_{t-1}) + \eta_t
\end{displaymath}
where $\alpha_t$ is the state at time $t$ and 
the function $T$ embodies the dynamics mediating the
development of the state at time $t$ from the state at the previous
time with random process error, $\eta_t$.

The transition model is constructed from finite difference
approximations of equation (\ref{eqn:sSIR}) with associated log-normal
random errors.
\begin{eqnarray}
\label{eqn:sSIRfd}
I_t &=& I_{t-\Delta t}\big(1+\Delta t(\beta_{t-\Delta t} - \mu_{t-\Delta t}
- \gamma)\big)e^{\eta_t} ;\quad\eta\sim N(0,\sigma_\eta) \\ 
D_t &=& \big(D_{t-\Delta t} + \Delta t \mu_{t-\Delta t}I_{t-\Delta
t}\big)e^{\eta_t};\quad\eta\sim N(0,\sigma_\eta)
\end{eqnarray}

The general form of the state space {\bf observation model} is
\begin{displaymath}
x_t = O(\alpha_t) + \varepsilon_t1G
\end{displaymath}
where the function $O$ describes the measurement process with
error $\varepsilon$ in observing the state.

The observed time series for $I$ and $D$ begins at the first recorded
case. The first recorded death occurs several days or weeks after the
first recorded case and therefor may have a considerable number of
recorded zeros. I applied two different error models for cases and
deaths. The observation model for cases is a simple log-normal error
\begin{equation}
\log\varphi_t = \bigg(\log\frac{1}{\sqrt{2\pi\sigma^2_I}} -\Bigl(\frac{\log
I_t-\log\widehat{I}_t}{\sigma_I}\Bigr)^2\bigg)\\
\end{equation}

The observation model for deaths is assumed to be ``zero-inflated''
log normal likelihood given by
\begin{equation}
  \log \varepsilon_t = \left\{
    \begin{array}{r@{\;:\quad}l}
       D_t > 0 &
(1-p_0)\cdot\bigg(\log\frac{1}{\sqrt{2\pi\sigma^2_D}}
          -\Bigl(\frac{\log D_t-\log\widehat{D}_t}{\sigma_D}\Bigr)^2\bigg)\\
       D_t = 0 & p_0 \cdot\log \frac{1}{\sqrt{2\pi\sigma^2_Y}}\\
    \end{array}
  \right.
\end{equation}
where $p_0$ is the proportion of observed deaths observations equal to zero.

\begin{table}[b]
\caption{Complete list of estimated and computed parameters and model
constraints for the simple SIR model.
There are ?? estimated parameters. 
The ?? computed variables are functions of estimated parameters.
There are ?? constraints and constants.
There are two random effects for each time step in the model.
}
\label{tab:allvars1}
\begin{center}
\begin{tabular}{ll}
\hline
Parameter & Definition\\
\hline
\hline
       & {\it Estimated parameters:}\\
$\sigma_I$ & Infectious compartment estimation standard deviation\\
$\sigma_D$ & Deaths compartment estimation standard deviation\\
$\sigma_\eta$ & Standard deviation of Infectious and Deaths compartments\\
$\sigma_\beta$ & Standard deviation of infection rate random walk\\
$\sigma_\mu$ & Standard deviation of mortality rate random walk\\
       & {\it Random effects:}\\
$\beta_t$ & Infection rate\\
$\mu_t$ & Mortality rate\\
       & {\it Fixed parameters:}\\
$\sigma_I$ & Infectious compartment estimation standard deviation\\
$\sigma_D$ & Deaths compartment estimation standard deviation\\
\hline
\end{tabular}
\end{center}
\end{table}


\end{document}

