
\documentclass[12pt,letterpaper]{article}
\usepackage{amsmath}
\usepackage{amsfonts}
\usepackage[figurewithin=section,tablewithin=section]{caption}
\usepackage[usenames,dvipsnames]{color}
\usepackage{graphicx}
\usepackage{longtable}
\usepackage{rotating}
\usepackage{booktabs}
\usepackage[citestyle=authoryear,bibstyle=authoryear,maxbibnames=99]{biblatex}
%\addbibresource{~/Projects/MyBibtex.bib}
\bibliography{/home/jsibert/Projects/MyBibtex.bib}


\usepackage[pdftex,bookmarks=false]{hyperref}
\hypersetup{pdfauthor={John Sibert}
            pdfsubject={variable infection and mortality rate estimates}
            pdftitle={Simple SIR Statistical Model}
            pdfkeywords={COVID-19, SIR Model, random effects, TMB}
            }%

\newcommand\doublespacing{\baselineskip=1.6\normalbaselineskip}
\newcommand\singlespacing{\baselineskip=1.0\normalbaselineskip}

\title{Estimating short term trends in infection and mortality rates
during the Covid 19 Epidemic}

\author{
John Sibert\thanks{johnrsibert@gmail.com}\\
Joint Institute of Marine and Atmospheric Research\\
University of Hawai`i at M\={a}noa\\
Honolulu, HI  96822 U.S.A.\\[0.125in]
\date{\today}
}

\begin{document}

\maketitle

\doublespacing

\section*{Introduction}

The sudden advent of the COVID-19 pandemic provoked many political
jurisdictions to advise people to ``shelter in place'' and to practice
''social distancing''. If this advice has been effective it should be
possible to detect the effects of the advice by examining changes
numbers of infected people and perhaps changes in the infection rates
between areas. The SIR models of epidemic spread divide the effected
population into three compartments: 
Susceptible, Infectious and Recovered.
SIR models are
usually expressed as coupled ordinary differential equations,
\begin{eqnarray}
\label{eqn:SIR}
\frac{dS}{dt} &=& -\beta\frac{IS}{N} - \mu S\\
\frac{dI}{dt} &=& \hphantom{-}\beta\frac{IS}{N} - \mu I -\gamma I\\
\frac{dR}{dt} &=& \hphantom{-}\gamma I - \mu R 
\end{eqnarray}
where $N$ is the population size, $\beta$ is the instantaneous
infection rate ($[t^{-1}]$), $\mu$ is the instantaneous mortality rate
($[t^{-1}]$),  and $\gamma$ is the instantaneous recovery rate
($[t^{-1}]$).  

Unfortunately, few data sets include data from each of
these compartments. 
The New York Times ``historical'' data
(https://github.com/nytimes/covid-19-data/) an
easily accessible and source of data. These
data comprise daily totals of ``cases'' and ''deaths'' for each county
in the United States. I assume that the data included as ``cases'' are
a reasonable approximations of the Infectious compartment ($I$) in a SIR model. 
There is simply no credible information of either Susceptible or
Recovered compartments.

So I made some simplifying assumptions. (1) The entire population is
susceptible so that $S/N = 1$. (2) Over the short term, the size of the
Susceptible compartment does not change, $\frac{dS}{dt} = 0$.
(3) People who recover from a COVID-19 infection return to the Susceptible
compartment, eliminating the Recovered compartment. 
With these assumptions, and the addition of a ``dead''
compartment the simplified SIR model is
\begin{eqnarray}
\label{eqn:sSIR}
\frac{dI}{dt} &=&  \beta I - \mu I -\gamma I\\
\frac{dD}{dt} &=& \mu I
\end{eqnarray}
and has state variables that might be fitted to available observations.

The data available during the initial stages of the COVID-19 pandemic
contain measurement errors of various types.
Definitions and methods of detecting and reporting the numbers of
infected persons vary between political jurisdictions (or
``geographies'' in the parlance of the New York Times) and may also
change with time.
Comparable uncertainties also occur in reporting of deaths caused
by COVID-19 infection.
There is also variability in the biosocial
processes that mediate disease transmission.

State-space models separate variability in the biosocial
processes in the system (transition model)
from errors in observing features of interest
in the system (observation model).
(See \cite{Harvey1990}).

The general form of the transition model is
\begin{displaymath}
\alpha_t=T(\alpha_{t-1}) + \eta_t
\end{displaymath}
where $\alpha_t$ is the state at time $t$ and 
the function $T$ embodies the dynamics mediating the
development of the state at time $t$ from the state at the previous
time with random process error, $\eta_t$.

The transition model is constructed from finite difference
approximations of equation (\ref{eqn:sSIR}) with associated log-normal
random errors.
\begin{eqnarray}
\label{eqn:sSIRfd}
I_t &=& I_{t-\Delta t}\big(1+\Delta t(\beta_{t-\Delta t} - \mu_{t-\Delta t}
- \gamma)\big)e^{\eta_t}\\
D_t &=& \big(D_{t-\Delta t} + \Delta t \mu_{t-\Delta t}I_{t-\Delta
t}\big)e^{\eta_t}
\end{eqnarray}
where $\eta$ is a log-normal random deviate, $\eta\sim
N(0,\sigma_\eta)$, representing temporal variability in the biosocial
factors that mediate the spread of the pandemic. I have no particular
justification, beyond the parsimony principle, for the assumption that
the variance of these two processes, $\sigma_\eta$, should be the
same.

One approach to modeling time-dependent rates of infection and
mortality, $\beta$ and $\mu$, is to assume that they are both
log-normal random walks. I assume that
\begin{eqnarray}
\log\beta_t &=& \log\beta_{t-\Delta t}+\varepsilon;\quad \varepsilon\sim 
N(0,\sigma_\beta)\\
\log\mu_t &=& \log\mu_{t-\Delta t}+\varrho;\quad \varrho\sim
N(0,\sigma_\mu)
\end{eqnarray}

The general form of the state space observation model is
\begin{displaymath}
x_t = O(\alpha_t) + \varepsilon_t
\end{displaymath}
where the function $O$ describes the measurement process with
error $\varepsilon$ in observing the state.

I applied two different error models for cases and
deaths. The observation model for cases is a simple log-normal error
\begin{equation}
\log\varphi_t = \bigg(\log\frac{1}{\sqrt{2\pi\sigma^2_I}} -\Bigl(\frac{\log
I_t-\log\widehat{I}_t}{\sigma_I}\Bigr)^2\bigg)\\
\end{equation}
where $I$ is the observed number of cases and $\widehat{I}$ is the
number of cases predicted by equation~\ref{eqn:sSIRfd}.

The deaths time-series inevitably contains a
substantial number of recorded zeros. 
Not all those afflicted by COVID-19 have died; there are far fewer
deaths than infections. In addition,
the observed time series for both $I$ and $D$ begins at the first recorded
case. The first recorded death occurs several days or weeks after the
first recorded case. 
The observation model for deaths accommodates observed zeroes by
assuming to be ``zero-inflated'' log normal likelihood given by
\begin{equation}
  \log \varepsilon_t = \left\{
    \begin{array}{r@{\;:\quad}l}
       D_t > 0 &
(1-p_0)\cdot\bigg(\log\frac{1}{\sqrt{2\pi\sigma^2_D}}
          -\Bigl(\frac{\log D_t-\log\widehat{D}_t}{\sigma_D}\Bigr)^2\bigg)\\
       D_t = 0 & p_0 \cdot\log \frac{1}{\sqrt{2\pi\sigma^2_D}}\\
    \end{array}
  \right.
\end{equation}
where $p_0$ is the proportion of observed deaths observations equal to zero,
$D$ is the observed number of deaths and $\widehat{D}$ is the
number of deaths predicted by equation~\ref{eqn:sSIRfd}.

\begin{table}[b]
\caption{Complete list of estimated and computed parameters and model
constraints for the simple SIR model.
There are ?? estimated parameters. 
The ?? computed variables are functions of estimated parameters.
There are ?? constraints and constants.
There are two random effects for each time step in the model.
}
\label{tab:allvars1}
\begin{center}
\begin{tabular}{ll}
\hline
Parameter & Definition\\
\hline
\hline
       & {\it Estimated parameters:}\\
$\sigma_I$ & Infectious compartment estimation standard deviation\\
$\sigma_D$ & Deaths compartment estimation standard deviation\\
$\sigma_\eta$ & Standard deviation of Infectious and Deaths compartments\\
$\sigma_\beta$ & Standard deviation of infection rate random walk\\
$\sigma_\mu$ & Standard deviation of mortality rate random walk\\
       & {\it Random effects:}\\
$\beta_t$ & Infection rate\\
$\mu_t$ & Mortality rate\\
\hline
\end{tabular}
\end{center}
\end{table}

\printbibliography[title=References]

\end{document}

